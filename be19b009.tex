
\boldmath {$$ \int {kq \over r^2} $$}

Coulomb's law, or Coulomb's inverse-square law, is an experimental law of physics that quantifies the amount of force between two stationary, electrically charged particles. The electric force between charged bodies at rest is conventionally called electrostatic force or Coulomb force. Although the law was known earlier, it was first published in 1785 by French physicist Charles-Augustin de Coulomb, hence the name. 

The law states that the magnitude of the electrostatic force of attraction or repulsion between two point charges is directly proportional to the product of the magnitudes of charges and inversely proportional to the square of the distance between them, 

